\section{Introduction}
\label{sect:intro}
In the last 50 years, evidence has steadily mounted, that the constituents and conditions necessary for life are common in the Universe \citep{Wordsworth2014}. Predicting specific properties of electromagnetic emissions from extraterrestrial technologies is one of the most challenging aspects of searching for life in the universe. However, it also represents a high-risk, high-reward endeavor. If an extraterrestrial civilization were intentionally attempting to indicate its presence through such emissions, it would be advantageous to make the signals easily distinguishable from natural phenomena. The evidence of such emissions is referred to as `technosignatures', and the field dedicated to their detection is known as the Search for Extra-terrestrial Intelligence (SETI).


It is commonly assumed that civilizations elsewhere in the universe may employ similar technologies to those developed on Earth. Consequently, radio frequencies are considered a logical domain for conducting SETI surveys due to the widespread use of telecommunications and radar. Therefore, radio astronomy has played a significant role in the field of SETI since the 1960s \citep{Drake:1961bv, Tarter:1980p1516}. Numerous previous SETI surveys have utilized large single dish telescopes operating at frequencies $\gtrsim$ 1\,GHz \citep{Tarter:1996jf, Siemion_KEPLER_ApJ, Enriquez:2017} \footnote{See \citealt{wright18} for a review}. However, exploration of the radio window below 1\,GHz has been relatively limited. Technosignature searches commonly seek narrowband (approximately Hz-scale) radio emissions, either transmitted directly or leaking from other civilizations. Nonetheless, there is no inherent preference for any specific segment of the radio spectrum, which necessitates surveys spanning from low frequencies (30\,MHz) to high frequencies (100\,GHz; \citealt{Cherry_2022}). At 30\,MHz it becomes very difficult to observe from the ground due to ionospheric conditions \citep[see][chap.~7.8]{burke_graham-smith_wilkinson_2019}. This study primarily focuses on low-frequency SETI in the 110 - 190\,MHz range.

\subsection{Scientific Motivation}
Low-frequency radio SETI presents significant challenges due to higher sky temperatures, which limit the sensitivity of the underlying observations. The Murchison Widefield Array (MWA, \citealt{2013PASA...30....7T}) in Western Australia has been at the forefront of low-frequency SETI research thus far \citep{tingay:center,tingay:anticenter,tingay:omm}. However, the LOw Frequency ARray (LOFAR) presents a compelling scientific case for conducting a SETI survey \citep{Eavesdropping}. Aside from operating at low frequencies in the northern sky, LOFAR offers a large field of view, enabling the search for technosignatures across thousands of stars in each observation. 

Radio SETI also grapples with the challenge of handling a significant amount of radio frequency interference (RFI). Traditionally, SETI surveys have been conducted using single dish radio telescopes. While these telescopes offer operational convenience and room for upgrades, they possess limitations in effectively distinguishing between sources of interference and authentic sky-bound signals unless equipped with multibeam receivers. In contrast, the utilization of two local LOFAR stations presents two notable advantages over conventional single dish surveys. As demonstrated in studies conducted by \cite{Enriquez:2017} and \cite{danny-parkes}, single dish surveys typically employ an 'ON' and 'OFF' observing technique, where the target is observed for five minutes ('ON') followed by five minutes of observing a different location ('OFF'). This cycle is repeated three times, resulting in three 'ON' pointings and three 'OFF' pointings. This approach facilitates the identification and elimination of narrowband signals detected in the local environment that could potentially interfere with the search for technosignature candidates. By employing multiple stations, the search benefits from the unique local RFI environments at each station. This leads to a higher rate of rejecting false positive signals compared to the aforementioned surveys, which have signals of interest on the order of thousands. Additionally, since there are two stations involved, there is no requirement to alternate between an 'ON' and 'OFF' observation regime. As a result, the entire observation duration can be dedicated to directly observing the target, as the comparison of RFI environments would yield the same effect. This characteristic renders SETI surveys with two or more telescopes a highly valuable resource, particularly in today's RFI environments.

\subsection{Breakthrough Listen}
The Breakthrough Listen (BL) program is conducting one of the most comprehensive searches for evidence of intelligent life by extending the search to a wide variety of targets from existing ground-based observing facilities (see \citealt{gsc+19} for a review). All of the existing observations within the BL program have so far been conducted in the $1-27.45$~GHz range \citep{Cherry_2022}. \\ 
In this paper we report on low frequency extension of the BL initiative using two international LOFAR stations to perform simultaneous dual-site observations of nearby exoplanet candidates of interest from both the Transiting Exoplanet Survey Satellite \citep[TESS;][]{TESS_2015} and from \textit{Gaia} \citep{Gaia_DR2_HR_2018} in collaboration with the BL program. This survey also demonstrates the proof of concept of using dual-site observations for the rejection of spurious local sources of terrestrial origin. This method thus removes the need for separate `ON' and `OFF' pointings \citep{Gajjar_2021_BLGC1}. In \S~\ref{sec:observations} we describe the observational set up and the data acquired. \S~\ref{sect:signal_types} explains the data analysis steps taken. We discuss the implications of this work in \S~\ref{sec:discussion} before concluding in \S~\ref{sec:conclusion}.