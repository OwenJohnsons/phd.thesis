\section{Conclusion}\label{sec:conclusion}

This paper presents a SETI search in a mostly unexplored parameter space as seen with other SETI surveys \citep{Gajjar_2021_BLGC1, Price:2018bv} by simultaneously observing TESS and Gaia targets of interest in the $110-190$~MHz radio window. It also demonstrates dual-site coincidence rejection showing that it provides a new method to discriminate candidate extra-terrestrial signals from terrestrial radio frequency interference. We propose this method as a promising means of follow-up for confirmation of any candidates interest arising in this type of study or others in this frequency range. Each target within our fields was then searched for narrow-band signals at each station using our most up to date search techniques \citep{Enriquez:2017}. The benefit of barycentric correction for eliminating false positives is also demonstrated. Finally the first stringent constraints on the fraction of transmitting civilizations at this frequency range have been shown further contraining the parameter space the Drake equation presents.   

As the LOFAR SETI observation campaign continues and more high resolution frequency data is collected, a machine learning search method comparable to \cite{BLSETI_ML_2023} can be trained and implemented to seek out signals of interest. Multiple LOFAR stations, or indeed sub-arrays of any other wide-footprint radio array allows the option for a coincidence rejection method over the `ON' and `OFF' beam pointings used previously. For future low-frequency SETI surveys, the use of further international stations and a prolonged observation campaign will place even further constraints on an ETI residing in this parameter space. The addition of one or more LOFAR stations would allow for the use of localising a signal of interest in the u-v plane which would be useful for follow up observations. This would be done post-facto through correlation of saved voltages for any candidates of interest.
