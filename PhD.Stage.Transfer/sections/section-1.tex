\setcounter{page}{1} 
\pagenumbering{arabic}
\section{A Prelude to Pulsars} \label{sec:prelude}
When stars with a mass of at least $8 \smass$ reach the end of their evolutionary stage they experience a depletion of nuclear fuel will and undergo a core collapse.  This results in the star exploding as a supernova. Depending on the mass of the host star the Supernova will form a black Hole or a neutron Star. Based on the electron degeneracy pressure limit \citep[pp. 434 -- 443]{chandrasekhar_introduction_2012} stars that fall in the range of 20 - 30 $\smass$ form neutron stars \citep{heger_how_2003}. \

Neutron stars are supported against further collapse by the presence of neutron degeneracy pressure which arises from the Pauli exclusion principle. Strong nuclear forces between the neutrons also provides additional support against gravitational collapse. With these three opposing forces a stable equilibrium is formed \citep{1983bhwd.book.....S}. \

In turn, this makes neutron stars exceptionally dense, they are the densest known objects in the universe that emit light. The average density of a neutron star is $10^{17} \text{kg/m}^3$ \citep{baym_neutron_1971} and their radii are comparable to the size of cities, with radii of 10 - 20 km. \

During collapse the conservation of magnetic flux plays a crucial role in the large strength magnetic fields that are observed in neutron stars along with contributions from the dynamo effect and frozen-in magnetic fields. The strength of a pulsar's magnetic field is on the order of $10^{12}$ - $10^{15}$ G \citep{michel_theory_1982}. \

Charged particles accelerate along the magnetic field lines in the magnetosphere of the neutron star. These particles emit electromagnetic radiation in a cone shape along the magnetic axis. If the magnetic axis is not aligned with the rotational axis of the neutron star, the radiation beam will sweep across the sky. This is known as a pulsar, a Galactic lighthouse.% analogus to cosmic lighthouses. 

\subsection{The Population of Pulsars}

At the time of writing, there are currently more than 3380 known pulsars. Since their discovery by Jocelyn Bell Burnell \citep{hewish_observation_1968}, the population has grown immensely, but there remain many open questions about pulsar evolution and the subclasses that lie within the population as a whole. Similarly to how exoplanet populations are shown using the mass-radius diagram and stellar populations are shown using the Hertzsprung-Russell diagram, pulsar populations are shown using what is known as the $P-\dot P$ diagram.

$P$ represents the pulsar's rotational period and $\dot P$ its derivative. These are key ways that pulsars are classified and studied in the context of their evolution. An example of a $P-\dot P$ diagram is shown in \cref{fig:p-pdot}. Different values on the plot indicate roughly the pulsar's age and magnetic field strength. \Cref{fig:p-pdot} shows the vastly different values between pulsars in the millisecond range and pulsars in the second range.

Theoretically, it has been shown that pulsars can be related to a ``death" line in the $P-\dot P$ diagram. This is the line where pulsars are no longer able to emit radio waves due to the pulsar's magnetic field no longer being strong enough to accelerate particles along the magnetic field lines. However, it has been shown that pulsars do exist below this line. The area below this line is commonly referred to as the ``graveyard".
\begin{figure}
    \centering
    \includegraphics[width=0.8\textwidth]{figs/PPdot-diagram.pdf}
    \caption{The $P-\dot P$ diagram showing the population of pulsars. The the millisecond pulsar sub-classes are colour coded. The red region represents the death line, where pulsars are theoretically no longer able to emit radio waves.}
    \label{fig:p-pdot}
\end{figure}

\subsection{The Properties of Pulsars}

The following section gives a brief non-exhaustive overview of some of the key properties of pulsars. 

\subsubsection{Neutron Star Radius \& Mass}

Understanding the mass of pulsars is important for understanding their evolution and equation of state. \cite{oppenheimer_massive_1939} derived a canonical mass limit of neutron stars to be 1.4 $\mathrm{M_{\odot}}$, but experimentally, this has been shown to be higher, with the largest mass of a pulsar observed to be $\sim 2.35 \mathrm{M_{\odot}}$ \citep{Romani_2022}. The mass-radius relationship of a pulsar is defined by an equation of state and a maximum mass limit. Redshifts and gravitational effects observed in pulsars exhibit the observed temperature and flux to be smaller than the actual value. The observed radius $R_\text{obs}$ can be described as follows \citep[p. 56][]{pulsar_handbook}:

\begin{equation}
    R_\text{obs} = \frac{R}{\sqrt{1 - \frac{2GM}{Rc^2}}} = \frac{R}{ \sqrt{1 - \dfrac{R_s}{R}}}
\end{equation}

where $R$ is the pulsar's radius and $M$ is the gravitational mass, $G$ is the gravitational constant, $c$ is the speed of light and $R_s$ is the Schwarzschild radius. \
The lower limit of the neutron star radius is described by \cite[p. 58][]{pulsar_handbook}: 

\begin{equation}
    R_\text{min} \simeq 1.5 \ R_s = \frac{3GM}{c^2} = 6.2 \ \text{km.} \left( \frac{M}{1.4 \smass} \right)
\end{equation}

Opposite to this the upper limit of the radius is obtained by requiring that there is stability against breaking up due to centrifugal forces. This gives \cref{eq:radius-max} following as described in \cite[p.~58]{pulsar_handbook}. 

\begin{equation}
    R_\text{max} \simeq \left(\frac{GMP^2}{4 \pi^2}\right)^{1/3} = 16.8 \ \text{km} \left( \frac{M}{1.4 \smass} \right)^{1/3} \left( \frac{P}{\text{ms}} \right)^{2/3}
    \label{eq:radius-max}
\end{equation}

Most pulsars are theoretically thought to have radii in the range of 10 - 15 km \citep{lattimer_neutron_2001}, giving them the unique position of `almost' black holes. 

\subsubsection{Spin Evolution}
One of the most unique characteristics of pulsars is the spinning that they exhibit. Understanding the spin evolution gives insight into many parameters of the pulsars, most notably the stage of their evolution. Pulsars begin their life in the upper end of the $P-\dot P$ diagram and slowly move down and to the right as they age due to a loss in rotational energy, commonly referred to as spin-down luminosity. The spin-down ($\dot E$) is described as follows \citep[p.~59]{pulsar_handbook}:

\begin{equation}
\dot E = - \frac{dE_{\text{rot}}}{dt} = 4\pi^2 I \dot P P^{-3}
\label{eq:spin-down-energy}
\end{equation}

Where $I$ is the moment of inertia. It is important to note that the energy loss that is converted into radio emission is almost negligible in comparison to the total energy loss from spin down. 

% \subsubsection{Radio Emission Beam}

% Pulse width

% \begin{equation}
%     \cos \rho = \cos \alpha \cos (\alpha + \beta) + \sin \alpha \sin (\alpha + \beta) \cos \left( \frac{W}{2} \right)
% \end{equation}

\subsubsection{Braking Index}

Pulsars have strong magnetic dipoles. According to classical mechanics, a rotating magnetic dipole that exhibits a moment, $|m|$, emits an electromagnetic wave at the pulsar's rotation frequency \citep[p.~60]{pulsar_handbook}. The dipole's radiation power is characterized by:

\begin{equation}
    \dot E_{\text{dipole}} = \frac{2}{3c^3} |m|^2 \omega^4 \sin^2 \alpha
\end{equation}

Where $\alpha$ is the angle between the magnetic axis and the rotation axis. Equating the above equation to the loss of rotational energy described in \cref{eq:spin-down-energy} gives the following for the expected evolution of the period:

\begin{equation}
    \dot \Omega = - \frac{2}{3Ic^3} |m|^2 \Omega^3 \sin^2 \alpha
\end{equation}

This is more commonly written as a power law, 

\begin{equation}
    \dot \nu = -K \nu^n 
    \label{eq:spin-down-power-law}
\end{equation}

\Cref{eq:spin-down-power-law} in terms of the period is $\dot P = K P^{2-n}$. Since this is a first-order differential equation, the solution can be integrated and given a constant, $K$, which provides an expression of age:

\begin{equation}
    T = \frac{P}{(n-1)\dot P} \left\{ 1 - \left( \frac{P_0}{P} \right)^{n - 1} \right\} 
    \label{eq:age-full}
\end{equation}

Here $P_0$ is the initial period of the pulsar. Commonly an assumption is made that the current period is much greater than the initial period $(P_0 \ll P)$. If it is also assumed that the pulsar is spinning down to due dipole magnetic radiation ($n=3$), \cref{eq:age-full} can be simplified into a characteristic age. 

\begin{equation}
    \tau_c \cong 15.8~\text{Myr} \left( \frac{P}{s} \right) \left( \frac{\dot P}{10^{-15}} \right)^{-1}
\end{equation}

The above estimation for a pulsars age is known to be inconsistent with theory to varying degrees. In cases where a Supernovae has been observed and has produced a pulsar the age is known to a much higher degree of accuracy. The Crab pulsar is one such example with an observed Supernova event in 1054 AD by Chinese astronomers \citep{kaspi_chandra_2001}. %Supernoave remnants are also used to estimate the age of pulsars. 

\subsubsection{Dispersion Measure} \label{sec:DispersionMeasure}

The interstellar medium (ISM) is a complex mixture of gas, dust and magnetic fields that fills the space between stars in a galaxy. Given that the ISM is a cold and ionised plasma any electromagnetic radiation will undergo a frequency-dependant index of refraction as they propagate. The following equation describes the refractive index of the ISM neglecting Galactic magnetic field \citep[p.~85]{pulsar_handbook},

\begin{equation}
    \mu = \sqrt{1 - \left( \frac{f_p}{f} \right)^2}
    \label{eq: ism-refractive-index}
\end{equation}

Where $f_p$ is the plasma frequency, $8.5 ~\text{kHz} \left(n_e/\text{cm}^{-3} \right)^{1/2}$ and $f$ is the frequency of the observed radiation. \\ If the refractive index of the ISM $\mu < 1$ then it can be assumed that the group velocity of the radiation is $v_g = c\mu$ which is sub light speed. The path of radiation from a pulsar to the observer will be delayed in time with respect to an infinite frequency by an amount:

\begin{equation}
    t = \left( \int^d_0 \frac{dl}{v_g} \right) - \frac{d}{c}
\end{equation}

If is assumed to be $f_p \ll f$, $\mu$ can be approximated.

\begin{equation}
    t = \frac{1}{c} \int^d_0 \left(1 + \frac{f_p^2}{2f^2} \right) dl - \frac{d}{c}  = \frac{e^2}{2 \pi m_e c} \dfrac{\int^d_0 n_e ~dl}{f^2} \equiv \mathcal{D} \cdot \dfrac{\text{DM}}{f^2}
\end{equation}

Where $\mathcal{D}$ is the dispersion constant and $\text{DM}$ is the dispersion measure. Each are commonly expressed as follows, $\mathcal{D}= 4.15 \times 10^3 ~\text{MHz}^2 ~\text{pc}^{-1} ~ \text{cm}^3 ~\text{s}$ and $\text{DM} = \int^d_0 n_e ~dl ~\text{cm}^{-3} ~\text{pc}$. This definition was adapted from \citet[p.~86]{pulsar_handbook} and \cite{taylor_recent_1977}.

% \subsection{Pulsar Subclasses}
% Following breif overview of the properities of pulsars, this section will give a breif overview of the subclasses of pulsars. The population of pulsars can be broken down into subclasses based on unique patterns in their properities. The main subclasses\footnote{This is a non-exhaustive list.} are as follows:

% \begin{enumerate}
%     \item Normal Pulsars: These are the most common type of pulsars. They are characterized by their regular pulses and are often observed in radio wavelengths. They are also known as radio pulsars. %they cover central region of the $P-\dot P$ diagram.

%     \item Rotating Radio Transients (RRATs): These are a subclass of pulsars that were initially discovered through their sporadic radio bursts rather than regular pulses. They exhibit irregular and infrequent radio emission.

%     \item Magnetars: While not exclusively pulsars, magnetars are highly-magnetized neutron stars that can also emit pulsed radiation. They are characterized by extremely strong magnetic fields, much more intense than typical pulsars.

%     \item Binary Pulsars: These are pulsars that are in orbit around another star, usually a normal (non-neutron) star. The interaction with the companion star can have significant effects on the pulsar's behavior.
    
%     \item Millisecond Pulsars (MSPs): These are pulsars with very short rotation periods, typically less than 10 milliseconds. They are believed to be old pulsars that have been spun up by the accretion of mass from a companion star in a binary system.

%     \item X-ray Pulsars: Pulsars that emit pulsed X-ray radiation fall into this category. These pulsars are typically observed in binary systems where the pulsar accretes matter from its companion star, leading to X-ray emission.

%     \item Anomalous X-ray Pulsars (AXPs) and Soft Gamma-ray Repeaters (SGRs): These are closely related to magnetars and are characterized by their intense and variable X-ray and gamma-ray emission. They are believed to be neutron stars with extremely strong magnetic fields.
% \end{enumerate}

\subsection{Spider Pulsars}

One type of pulsar that is of interest to this project is a subclass of transitional millisecond pulsars known as spider pulsars. Spider pulsars fall into two categories depending on their orbiting companion. The first category are known as Black Widow pulsars, segregated based on their companion mass falling in the range of 0.01 - 0.05 $\smass$ with a companion orbital period ($P_B$) of less than 10 hours. The second category are known as Redback pulsars and have a companion mass of 0.2 $\smass$ or greater with a $P_B$ of less than 1 day \citep{papitto_transitional_2022}.

\begin{figure}
\centering
\includegraphics[width=0.8\textwidth]{figs/theory/redback-multiwavelength-emission.jpg}
\caption{Figure taken from \cite{takata_multi-wavelength_2014}. Example of multiwavelength emission from a Redback pulsar.}
\end{figure}

It is thought that most millisecond pulsars are formed through the accretion of matter from an evolved compact binary system; the approximately 30\% found in isolation are thought to have ablated their companion star to the point of disassociation \citep{strader_optical_2019}. Material being thrown off the pulsar causes the radio emission to be eclipsed via scattering and absorption, for a segment of the companion's orbit. Redback systems exhibit both positive and negative period derivatives that are larger than the expected gravitational radiation and are thought to arise from the interaction of the companion's magnetic field and the pulsar's wind \citep{papitto_transitional_2022}.

Redback pulsars have been observed in two transitional states: ablation and accretion states \citep{papitto_transitional_2022}. These states are on sub-year timescales. The transition in stages sees the magnitude of optical emission increase by about one order of magnitude. Studying optical emission from redback pulsars informs on the heating of the companion, Roche-lobe filling fraction, and the mass of the system \citep{archibald_radio_2009, roberts_x-ray_2017}. Black widow pulsars have been observed to have little to no observable X-ray emission \citep{roberts_x-ray_2017}. However, redbacks have been shown to exhibit much more X-ray emission in their thermal spectra with consistent double peaks observable when the pulsar is at inferior conjunction \citep{roberts_x-ray_2017}. The observed companions of redbacks are mostly faint stars with temperatures around $2000$~K on the farside of the star from the pulsar \citep{breton_discovery_2013}. The companion's interaction with the pulsar dominates the thermal spectrum of spider pulsars from the present heating between the two. X-ray emission from redbacks shows hard X-ray spectra that follow a power law with photon indices ($\Gamma$) around 1 - 1.3. The energy of the thermal spectra in the X-ray is higher than what is expected from shock acceleration. Some models suggest a wind-wind shock between the pulsar and companion. However, this approach would require the wind momentum of the pulsar to be much weaker than the companion's. Similarly with the optical emission, the X-ray emission may be influenced by the magnetic field of the companion \citep{papitto_transitional_2022}. \ 

Redback's have also been known to emit $\gamma$-rays and is often the means in which Redback candidates are found. The primary source of gamma-ray emission in pulsar binaries is the pulsed magnetospheric radiation from the pulsar itself. Apart from a distinct dip observed in the light curves of eclipsing systems \citep{clark_neutron_2023, corbet_gamma-ray_2022}, it has been traditionally believed that gamma-ray emission remains consistent throughout the orbital cycle.

\subsection{Why study Redback Pulsars?}

Redback pulsars have a number of interesting science cases. Redbacks undergo a range of phenomena, including radio and X-ray pulsations, accretion processes, and periodic eclipses as the companion star passes in front of the pulsar. Their study also provides valuable insights into the evolution of binary systems, the behavior of pulsars, and the physics of accretion processes. Due to their transitional nature, they provide a glimpse into the evolution of pulsars in the latter stages of their life cycle. Pulsars are also used as tools to study theories of gravity, the interstellar medium, and probe for gravitational waves.

\subsection{Other exotic transients}

Redbacks themselves are exotic transients, but there are many other classes of radio exotic that are of interest to the community. In this project, work has also been carried out on an array of various radio transients and related objects. This includes the search for extraterrestrial intelligence (SETI), the study of M and Brown dwarf radio flares, and the probing of potential radio emission from exoplanets.

\subsubsection{Radio Stars and Exoplanets}

It has been well documented that our own star, the Sun, and planets in the solar system have radio emission associated with originating from auroras and stellar flares \citep{murphy_lofar_2021, zarka_auroral_1998}. With the advancement of modern radio telescopes, the prospect for capturing radio emission from others and even the prospect of detecting an exoplanet magnetosphere has become achievable \citep{vedantham_coherent_2020}. \

Observing stars especially at low frequency $(\leq 300 \ \text{MHz})$\footnote{$\lambda \sim 1 \ \text{m}$} act as probes into stellar and planetary plasma environments. Coronal Mass Ejections (CMEs) have a low-frequency burst component in which information about the kinematics of plasma can be deduced \citep{villadsen_ultra-wideband_2019}. Incident solar wind is also the primary driving force of auroral emission on magnetized planets. The goal of this project is to explore the viability of detecting radio emission from M and Brown dwarfs using I-LOFAR observations and archive data to put constraints on the prevalence of radio emission from these objects and explore the viability observing radio emission from exoplanets. \

Radio emission from stars is typically produced through CMEs, observed phenomenologically in the Sun as type II and III solar bursts. The radio emission from these events is thought to be produced through the electron-cyclotron maser instability \citep[ECMI;][]{EMI}. The ECMI is a plasma instability that occurs in the presence of a magnetic field and a population of energetic electrons. The instability is thought to be the primary driver of radio emission observed in type II bursts and stars as a whole. With type III bursts being thought to be the result of electron beams that are accelerated in the corona. \

Emission from planets has also been observed in the form of auroral emission. The most notable example of this is the Jovian system. The Jovian system is known to have auroral emission that is driven by the interaction of the solar wind with the magnetosphere of the planet. Coherent radio emission from the aurora is thought to be produced through the cyclotron maser instability \citep[ECM;][]{zarka_auroral_1998} which injects a high-velocity electron population into the magnetosphere. The maximum frequency of the ECM is directly proportional to the magnetic field strength of the object at its emitting point \citep{kavanagh_hunting_2023, joe_nature_review}.

Much time has been spent observing Ultracool Dwarfs ($\text{M7} <$) as they provide a good analogue to the Jovian systems. This allows for direct comparison between the two.  This provides valuable insights into the formation, atmospheric processes, and potential habitability of gas giants like Jupiter. Moreover, such comparative studies contribute significantly to our understanding of planetary evolution and diversity beyond our immediate neighbourhood. Recent detection of radiation belts around a UCD further supports the analogy to Jupiter, as radiation belts are a key characteristic of Jupiter's magnetosphere \citep{joe_nature_review}. \ UCDs, being intermediate in mass between stars and planets, serve as a bridge to understanding exoplanet detection. The discovery of bursting radio emission from a brown dwarf and subsequent detections in UCDs indicate departures from established stellar coronal/flaring relationships. \ Radio bursts from UCDs can exhibit periodic timing \citep{hallinan_rotational_2006} along with strong circular polarization and high brightness temperatures, suggesting the involvement of the ECM process in generating radio emission. Despite a significant amount of gigahertz-frequency radio searches, detection rates for UCD radio emissions remain stubbornly low at around 10\% overall \citep{lynch_radio_2016}.


\subsubsection{The Search for Extraterrestrial Intelligence} \label{sec:SETI-th}

The search for life elsewhere in the Universe has always been a burning question for many astronomers throughout history. Moreover, the prevalence of intelligent life in the universe has been a particularly intriguing aspect of this quest. Since the 1960s, astronomers have conducted consistent surveys, mainly in radio, aimed at detecting signals of artificial origin, commonly referred to as ``technosignatures''. These signatures are thought to resemble radio signals produced artificially on Earth and are mainly hypothesized to be narrowband drifting radio emissions leaking into space from transmitters. However, to date, there has been no positive detection of a technosignature. \ 

Similar to searches for dark matter, the lack of detection's has allowed researchers to place limits on the prevalence of civilizations in the galaxy. During the first radio Search for Extraterrestrial Intelligence (SETI) survey, \cite{Drake61Ozman} coined an equation to estimate the prevalence of civilizations in the local galaxy.

\begin{equation}
    N = R_* \cdot f_p \cdot n_e \cdot f_l \cdot f_i \cdot f_c \cdot L
    \label{DrakeEquation}
\end{equation}

Where $N$ is the number of civilizations in the galaxy, $R_*$ is the rate of star formation, $f_p$ is the fraction of stars that have planets, $n_e$ is the number of planets that could potentially support life, $f_l$ is the fraction of planets that develop life, $f_i$ is the fraction of planets that develop intelligent life, $f_c$ is the fraction of planets that develop technology, and $L$ is the lifetime of the civilization. However, in this endeavour there has been little to no SETI conducted at low frequencies in the radio regime as illustrated in \cref{fig:SETI-Frequency-Parameter-Space}. The main science goals of a low frequency technosignature search is to place tight constraints on the Drake equation at these frequencies, study methods on RFI mitigation and also use the high resolution data collected to search and study other radio transients, such as magnetars, Fast Radio Bursts (FRBs), pulsars and M-dwarfs.

\begin{figure}
    \centering
    \includegraphics[width=0.95\textwidth]{figs/SETI-Frequency-Parameter-Space.pdf}
    \caption{The radio frequency parameter space covered by technosignature searches as a function of sky coverage and sensitivity. The color coding represents the ability to detect an Arecibo-like planetary transmitter at various distances. Where $d_* \leq 25 \ \text{pc}$, $d_* \leq 75 \ \text{pc}$ and $d_* \leq 250 \ \text{pc}$ are shown in blue, green and red respectively. With anything beyond $250$~pc being shown in pink. The figure is adapted from \cite{ng_search_2022}.}
    \label{fig:SETI-Frequency-Parameter-Space}
\end{figure}


% \subsubsection{4FGL J0523-2529}

% \subsubsection{4FGL J2054-6904}